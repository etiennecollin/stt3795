\newcommand{\latexTemplatesPath}{/Users/etiennecollin/github/latex-templates/}

\documentclass[12pt]{article}
\usepackage[french]{babel}
\usepackage{\latexTemplatesPath/packages}
\usepackage[style=ieee,backref=true,backend=biber,date=iso,datezeros=true,seconds=true]{biblatex}

% DOCUMENT USER SETTINGS ==============================================================================================
\newcommand{\docAuthorName}{Etienne Collin | 20237904,\\Guillaume Genois | 20248507}
\newcommand{\docAuthorStudentNumber}{2038029}
\newcommand{\docAuthorTitlePage}{\docAuthorName}
\newcommand{\docClass}{Fondements Théoriques en Science des Données}
\newcommand{\docClassInstructor}{Stefan Horoi \& Guillaume Huguet}
\newcommand{\docClassNumber}{STT3795}
\newcommand{\docClassSection}{Section A}
\newcommand{\docClassSemester}{Hiver 2024}
\newcommand{\docDueDate}{13 Février 2024}
\newcommand{\docDueTime}{23:59}
\newcommand{\docSubtitle}{Proposition de Projet Final}
\newcommand{\docTitle}{Détection de Paraphrases}
\input{\latexTemplatesPath/templates/basic/page_settings}       % Imports custom page settings
\input{\latexTemplatesPath/templates/basic/environment}         % Imports custom environments and definitions
% \fancyhf[HR]{\docClassTime}                                   % Removes student number from right header

% SOURCE ==============================================================================================================
\begin{document}
\input{\latexTemplatesPath/templates/basic/title_page_udem}

% \todototoc
% \listoftodos
% \pagebreak

% START OF DOCUMENT ===================================================================================================
% \subfile{subfiles/...}

\section{Objectif}
L'objectif principal de ce projet est de développer
un système de classification de texte capable de déterminer si deux
phrases données sont des paraphrases l'une de l'autre. Nous nous
concentrerons sur l'utilisation d'un modèle Naive Bayes pour cette
tâche, en explorant également la possibilité de générer des paraphrases
si le temps le permet. En outre, nous évaluerons la généralité de notre
modèle en le testant sur le dataset GLUE.

\section{Ressources}
\begin{itemize}
	\item Le corpus \href{https://paperswithcode.com/dataset/mrpc}{MRPC} pour l'entraînement et l'évaluation du modèle.
	\item Dataset \href{https://paperswithcode.com/dataset/glue}{GLUE} pour tester la généralité du modèle.
	\item Outils de traitement du langage naturel (NLP) pour le prétraitement des données.
	\item Bibliothèques de machine learning en Python telles que scikit-learn pour la mise en œuvre du modèle Naive Bayes.
\end{itemize}

\section{Livraisons Attendues}
\begin{enumerate}
	\item Code source du modèle Naive Bayes pour la classification de paraphrases.
	\item Rapport détaillé sur l'approche utilisée, les résultats obtenus, et les éventuelles expérimentations sur la génération de paraphrases.
	\item Évaluation des performances du modèle sur le dataset GLUE avec une analyse approfondie des résultats.
	\item Présentation des conclusions et des pistes d'amélioration pour de futures recherches.
\end{enumerate}

\pagebreak
\section{Description}
Le projet se basera sur le corpus MRPC
(Microsoft Research Paraphrase Corpus), qui propose une collection de
paires de phrases annotées pour la détection de paraphrases. L'idée est
d'implémenter un classificateur Naive Bayes pour évaluer la similarité
sémantique entre les paires de phrases.


\begin{enumerate}
	\item \textbf{Classification de Paraphrases:}
	      Utiliser le modèle Naive Bayes pour la classification des paires de phrases en paraphrases
	      ou non-paraphrases. - Expérimenter avec différentes représentations de
	      texte telles que bag-of-words, TF-IDF, ou d'autres caractéristiques
	      pertinentes pour le modèle Naive Bayes.
	\item \textbf{Évaluation sur le Dataset GLUE (STS-B et QQP):}
	      \begin{itemize}
		      \item Tester la généralité du modèle en l'appliquant au dataset GLUE, qui
		            propose une diversité de tâches de compréhension du langage naturel.
		      \item Analyser les performances du modèle sur différentes tâches du GLUE
		            et tirer des conclusions sur sa capacité à généraliser.
	      \end{itemize}
	\item \textbf{Approfondissements (si le temps le permet)}
	      \begin{itemize}
		      \item Comparaison avec une implémentation SVM
		      \item Comparaison avec une implémentation random forest
		      \item Optimisation des performances
		      \item Génération de Paraphrases
		            \begin{itemize}
			            \item Si le temps le permet, explorer des méthodes de génération de
			                  paraphrases. Cela pourrait impliquer l'utilisation de techniques de
			                  réécriture automatique ou de modèles de génération de texte pour
			                  créer des variations sémantiques des phrases.
		            \end{itemize}
	      \end{itemize}
\end{enumerate}

\pagebreak
\section{Contributions}
Notre équipe de deux personnes collaborera étroitement tout au long
du projet en adoptant une approche de "pair-programming" et en
travaillant conjointement sur les différentes facettes du problème.
Étant donné que nous sommes confrontés à un domaine relativement
nouveau pour nous, nous prévoyons explorer collectivement les étapes
clés du projet.

Bien que nous prévoyions une répartition flexible des tâches, la
nature collaborative du projet signifie que chaque membre de l'équipe
participera activement à chaque phase, partageant les idées,
débattant des approches, et contribuant à la prise de décisions.
Cela permettra une compréhension approfondie du problème et une
expertise partagée au sein de l'équipe.

\begin{enumerate}
	\item \textbf{Exploration du Problème (Collaboratif)}
	\item \textbf{Prétraitement des Données (Pair-Programming)}
	\item \textbf{Implémentation du Modèle Naive Bayes (Pair-Programming)}
	\item \textbf{Évaluation sur le Dataset GLUE (Pair-Programming)}
	\item \textbf{Aprrofondissements:}
	      \begin{itemize}
		      \item Comparaison avec une implémentation SVM (Guillaume)
		      \item Comparaison avec une implémentation random forest (Guillaume)
		      \item Optimisation des performances (Etienne)
		      \item Génération de Paraphrases (Etienne)
	      \end{itemize}
	\item \textbf{Rapport (Collaboratif)}
\end{enumerate}
% END OF DOCUMENT =====================================================================================================
% % List of figures/tables
% \pagebreak
% \begin{appendix}
%     \phantomsection\listoffigures
%     \phantomsection\listoftables
% \end{appendix}

\pagebreak\phantomsection\printbibliography[heading=bibintoc]
%\nocite{}
\end{document}
